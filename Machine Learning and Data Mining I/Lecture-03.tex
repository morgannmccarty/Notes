\documentclass[12pt]{article}

\usepackage{hyperref}
\usepackage[margin=0.5in]{geometry}
\usepackage[fleqn]{amsmath}
\usepackage{listings}

\DeclareMathOperator*{\argmin}{argmin}

\title{
    Machine Learning and Data Mining I:\ Lecture 3\\
    Polynomial Regression and Regularization}
\author{Morgan McCarty}
\date{06 July 2023}

\begin{document}
    \maketitle

    \section{Polynomial Regression}
        \subsection{Linear Regression Review}
            \begin{itemize}
                \item Estimate the parameters of a regression function:
                \item $\hat{y}_i = w_0 + \Sigma_{j=1}^N {w_j}x_{ij}$ where:
                \begin{itemize}
                    \item $w = {({X^T}X)}^{-1}{X^T}y$ where:
                    \begin{itemize}
                        \item $w$ is the vector of weights
                        \item $X$ is the matrix of features
                        \item $y$ is the vector of observations
                    \end{itemize}
                    \item $y_i$ is the $i$th observation (of the vector of observations)
                    \item $x_{ij}$ is the $j$th feature of the $i$th observation
                    \item $N$ is the number of features
                \end{itemize}
                \item We find the solution by minimizing the squared error on the training data.
                \item We want the model to \underline{generalize} to new data (i.e.\ have low error on unseen [test] data).
                \item The parametric form of the regression function:
                \begin{itemize}
                    \item Linear in parameters
                    \item Can transform input features
                    \item Do we learn a straight line? Polynomial?
                \end{itemize}
            \end{itemize}
        \subsection{Polynomial Regression}
            \begin{itemize}
                \item Transform the input features to higher order polynomials.
                \item i.e.\ $f_w(x) = w_0 + w_1x + w_2x^2 + w_3x^3 + \cdots + {w_m}x^m$
                \item Here $m$ is the degree of the polynomial (the highest power of $x$).
                \item We can choose $m$, it is a hyperparameter.
                \item Does our optimization problem change? No, we can still use the same linear regression model, but we have to transform the input features.
                \item Our previous design matrix $X$ was:
                \begin{equation}
                    X = \begin{bmatrix}
                        1 & x_{11} & x_{12} & \cdots & x_{1p} \\
                        1 & x_{21} & x_{22} & \cdots & x_{2p} \\
                        \vdots & \vdots & \vdots & \ddots & \vdots \\
                        1 & x_{N1} & x_{N2} & \cdots & x_{Np}
                    \end{bmatrix}
                \end{equation}
                \item Where $N$ is the number of observations and $p$ is the number of features.
                \item All terms are linear in the parameters (i.e.\ $w_0, w_1, \ldots, w_N$ i.e.\ of order $1$).
                \item We can transform the input features to higher order polynomials:
                \begin{equation}
                    X = \begin{bmatrix}
                        1 & x_{11} & x_{11}^2 & \cdots & x_{11}^m \\
                        1 & x_{21} & x_{21}^2 & \cdots & x_{21}^m \\
                        \vdots & \vdots & \vdots & \ddots & \vdots \\
                        1 & x_{N1} & x_{N1}^2 & \cdots & x_{N1}^m
                    \end{bmatrix}
                \end{equation}
                \item Where $m$ is the degree of the polynomial and $N$ is the number of observations.
                \item The weight vector $w$ was previously:
                \begin{equation}
                    w = \begin{bmatrix}
                        w_0 \\
                        w_1 \\
                        \vdots \\
                        w_p
                    \end{bmatrix}
                \end{equation}
                \item Now it is:
                \begin{equation}
                    w = \begin{bmatrix}
                        w_0 \\
                        w_1 \\
                        \vdots \\
                        w_m
                    \end{bmatrix}
                \end{equation}
                \item The difference is that we have more features, but they are all linear in the parameters. That is that the parameters are still of order $1$ so we can still use linear regression.
                \item The observation vector $y$ remains the same:
                \begin{equation}
                    y = \begin{bmatrix}
                        y_1 \\
                        y_2 \\
                        \vdots \\
                        y_N
                    \end{bmatrix}
                \end{equation}
                \item We can still use the same linear regression model:
                \item We will minimize the mean squared error:
                \begin{equation}
                    \argmin_w \frac{1}{N} \Sigma_{i=1}^N {(y_i - f_w(x_i))}^2
                \end{equation}
            \end{itemize}
        \subsubsection{An Example}
            We have a training dataset sampled from the unit interval $[0, 1]$ with a target function $f(x) = 7.5\sin(2.5\pi{}x)$ 
            and gaussian noise $\epsilon_i \sim \mathcal{N}(0, \sigma^2)$.
            \begin{equation}
                y_i = f(x_i) + \epsilon_i
            \end{equation}
            The training dataset has $N$ observations. $(x_i, y_i)$ for $i = 1, \ldots, N$.
            \begin{enumerate}
                \item If we use Polynomial Regression with $m = 1$ (i.e.\ a straight line) the model fails to capture the underlying function. \\
                It has a massive training error and underfits the data. Meaning that the model is too simple to capture the underlying function.
                \item If we use Polynomial Regression with $m = 6$ (i.e.\ a polynomial of degree $6$) the model captures underlying function and the noise in the data. \\
                It has a negligible training error and overfits the data. Meaning that the model is too complex and fits to the noise, not just the function.
                \item If we use Polynomial Regression with $m = 3$ (i.e.\ a polynomial of degree $3$) the model captures underlying function without fitting to the noise in the data. \\
                It has a small training error and generalizes well to unseen data.
            \end{enumerate}
        \subsection{Overfitting}
            \begin{itemize}
                \item The model learns idiosyncrasies (noise) in the training data resulting in poor generalization.
                \item When does this happen?
                \begin{itemize}
                    \item There are more model parameters than training data points.
                    \item Model is too complex and fits to the noise in the training data.
                    \item E.g.\ a degree $N$ polynomial exactly fits $N+1$ data points.
                \end{itemize}
            \end{itemize}
        \subsection{Avoiding Overfitting}
            \begin{itemize}
                \item More training data: always works, but is not always possible in practice.
                \item Cross Validation: split the training data into a training set and a test set. Train the model on the training set and evaluate it on the test set.
                \item Regularization: Mathematical framework for controlling the complexity of the model.
            \end{itemize}
    \section{Regularization}
        \subsection{Ex. Polynomial Regression}
            More complex models (i.e.\ higher order or with larger $m$) lead to larger magnitude of model parameters.
            Slight perturbations in input features lead to large changes in the output.
            We can avoid overfitting by discouraging large model parameter values.
        \subsection{Occam's Razor}
            ``Among competing hypotheses, the one with the fewest assumptions should be selected''. \\
            In order to avoid overfitting, we should choose the simplest model that fits the data.
        \subsection{Regularized Linear Regression}
            \begin{itemize}
                \item Our original loss function:
                \begin{equation}
                    E(w) = \frac{1}{2} \Sigma_{i=1}^N {(y_i - {w^T}x_i)}^2
                \end{equation}
                \item Where $x_i$ is the $i$th observation and $y_i$ is the $i$th target value.
                \item We can penalize the weights to augment the error function (i.e.\ add a penalty/regularization term):
                \begin{equation}
                    E^{ridge}(w) = \frac{1}{2} \Sigma_{i=1}^N {(y_i - {w^T}x_i)}^2 + \frac{\lambda}{2} ||w||^2, \lambda \geq 0
                \end{equation}
                \item $||w||^2$ is the squared Euclidean norm (L2 norm) of the weight vector.
                \item E.g.\ for weight vector $w$:
                \begin{equation}
                    w = \begin{bmatrix}
                        w_0 \\
                        w_1 \\
                        \vdots \\
                        w_m
                    \end{bmatrix}
                \end{equation}
                \begin{equation}
                    ||w||^2 = w_0^2 + w_1^2 + \cdots + w_m^2
                \end{equation}
                \item $\lambda$ is the regularization parameter. It controls the strength of the regularization.
                \item $\lambda = 0$ is equivalent to the original loss function.
                \item $\lambda \rightarrow \infty$ is equivalent to setting all weights to $0$.
                \item We can also use the L1 norm:
                \begin{equation}
                    ||w||_1 = |w_0| + |w_1| + \cdots + |w_m|
                \end{equation}
                \item This is called Lasso Regression. Other norms are possible.
                \item This would require swapping the L2 norm for the L1 norm in the loss function.
                \item The current model of Regularized Linear Regression is also called Ridge Regression.
                \item It is important to note that the regularization term is only applied to the weights, not the bias term. This is because the bias term is not a function of the input features.
                \item As such the weight vector has $m + 1$ elements, but the regularization term only has $m$ elements.
                \item The training error will increase with the regularization parameter $\lambda$. This is because the model is penalized for having large weights.
            \end{itemize}
        \subsection{Gradient Descent with L2 Regularization}
            \begin{itemize}
                \item Once again we will use gradient descent to minimize the loss function.
                \item We will update the weights in two steps.
                \item First we update the intercept/bias term:
                \begin{equation}
                    w_0^{i+1} = w_0^i - \eta \Sigma_{j=1}^N (w_0^i + w^{iT}x_j-y_j)
                \end{equation}
                \item Notice how the regularization term is not included in the update rule for the bias term.
                \item Next we update the weights:
                \begin{equation}
                    w^{i+1} = w^i - \eta [ \Sigma_{j=1}^N (w_0^i + w^{iT}x_j-y_j)(x_j) + \lambda w^i ] \\
                    w^{i+1} = (1 - \eta \lambda) w^i - \eta \Sigma_{j=1}^N (w_0^i + w^{iT}x_j-y_j)(x_j) \text{, where } \lambda \geq 0
                \end{equation}
                \item The regularization term is included for the rest of the weights.
                \item The second equation corresponds to ``weight decay''.\ i.e.\ each weight is decayed by a factor of $(1 - \eta \lambda)$.
                \item ``Weight decay'' is a common term for regularization in neural networks. It is used because it is equivalent to L2 regularization. Both equations are not necessary to use together (i.e.\ you can use one or the other).
                \item As mentioned before, there are other norms that can be used for regularization.
                \item From a technical viewpoint: $\lambda$ is a rank conditioner for the data covarience matrix, which helps make sure that the matrix is invertible.
            \end{itemize}
    \subsection{Hyperparamter Learning}
        \subsection{Connection to What We've Learned}
            \begin{itemize}
                \item So far we have seen two hyperparameters:
                \begin{enumerate}
                    \item Learning Rate:\ $\eta$
                    \item Regularization Parameter:\ $\lambda$
                \end{enumerate}
                \item There are several ways to choose these hyperparameters:
                \begin{enumerate}
                    \item Grid Search
                    \item Random Search
                    \item Genetic Algorithms
                    \item Cross Validation
                \end{enumerate}
                \item These methods each have their own advantages and disadvantages.
            \end{itemize}
        \subsection{Methods}
            \begin{enumerate}
                \item Grid Search
                \begin{itemize}
                    \item Define a search space: i.e.\ a range of values for all hyperparameters of the model.
                    \item For Ridge Regression we have two hyperparameters: $\eta$ and $\lambda$.
                    \item We can set a range of values in $[0, 1]$ for our parameters and evenly divide the range into $100$ grid points.
                    \item At each value in the grid we train the model and estimate the performance metric (e.g.\ MSE, MAE, etc.).
                    \item Once we have evaluated the model at all grid points we choose the model with the best performance.
                    \item With many hyperparameters and a large range of values, this method can be very computationally expensive and time consuming. It may not be feasible to evaluate all grid points.
                \end{itemize}
                \item Random Search
                \begin{itemize}
                    \item Define a search space: i.e.\ a range of values for all hyperparameters of the model.
                    \item This functions similar to Grid Search, but instead of evaluating all grid points we randomly sample from the search space.
                    \item This is less computationally expensive than Grid Search, but it is unlikely to find the optimal hyperparameters.
                    \item It is possible to combine Grid Search and Random Search.
                    \item We can use Random Search to find a region of the search space that contains the optimal hyperparameters.
                    \item Then we can use Grid Search to find the optimal hyperparameters within that region.
                \end{itemize}
                \item Genetic Algorithms
                \begin{enumerate}
                    \item Define a search space: i.e.\ a range of values for all hyperparameters of the model.
                    \item This is similar to Random Search, but instead of randomly sampling from the search space we use a genetic algorithm to find the optimal hyperparameters.
                    \item This is less computationally expensive than Grid Search, but if the search space is large it will only approach the optimal hyperparameters.
                    \item It is possible to combine Grid Search and Genetic Algorithms (similar to Random Search).
                \end{enumerate}
                \item Cross Validation
                \begin{enumerate}
                    \item This is the most common method for choosing hyperparameters.
                    \item We will take our training data and split it into $k$ folds (a \underline{fold} is a subset of the training data).
                    \item $k$ is also a hyperparameter, but it is not the same as the hyperparameters of the model. Usually $k = 5$ or $k = 10$.
                    \item We will train the model on $k - 1$ folds and evaluate it on the remaining fold.
                    \item We will repeat this process $k$ times, each time using a different fold as the test set.
                    \item We will then average the performance metric (e.g.\ MSE, MAE, etc.) over the $k$ folds.
                    \item This is done at each point in the search space, so it is computationally expensive, but it is the most accurate method for choosing hyperparameters.
                    \item It is important to ensure that there is no ``leakage'' between the training and test sets as this will result in an overly optimistic performance metric (i.e.\ the model will overfit the data).
                    \item Over all the folds we will have $k$ performance metrics and then we will average them.
                \end{enumerate}
            \end{enumerate}
        \subsection{Final Notes}
            \begin{itemize}
                \item It is useful to standardize and normalize the input features before training the model.
                \item We can standardize the input features by subtracting the mean and dividing by the standard deviation.
                \item We can normalize the input features by subtracting the minimum and dividing by the range.
            \end{itemize}
\end{document}