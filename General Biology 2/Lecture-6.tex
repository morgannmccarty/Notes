\documentclass[12pt]{article}

\usepackage{amsmath}

% chktex-file 8

\title{
    General Biology 2: Lecture 6}
\author{Morgan McCarty}
\date{20 September 2023}

\begin{document}
    \maketitle

    \begin{itemize}
        \item Genes in Population
        \begin{itemize}
            \item Population genetics:\ study of genes and genotypes in a population
            \item Extend of genetic variation, why it exists, how it is maintained, how it changes over generations
            \item Helps to understand how genertic variation is related to phenotype variation
            \item Phenotypes exposed to outside pressures $\rightarrow$ genes are not
        \end{itemize}
        \item Genes in Natural populations
        \begin{itemize}
            \item Can be monomophic (single allele accounts for $99\%$) or polymorphic (multiple alleles)
            \item Polymorphism comes about through various changes
            \begin{enumerate}
                \item Duplication of gene region
                \item Deletion of significant region of gene
                \item Change in a single nucletide (SNP - single nucleotide polymorphism, smallest and most common change in a gene)
            \end{enumerate}
        \end{itemize}
        \item Allele Frequecy
        \begin{itemize}
            \item $= \frac{\text{Number of copies of an allele in a population}}{\text{Total number of all alleles for that gene in a population}}$
        \end{itemize}
        \item Genotype Frequency
        \begin{itemize}
            \item $= \frac{\text{Number of individuals with a particular genotype in a population}}{\text{Total number of individuals in a population}}$
        \end{itemize}
        \item Hardy-Weinberg Principle
        \begin{itemize}
            \item Genes remain in equilibrium and allele frequencies remain constant from generation to generation unless acted upon by outside influences
            \item Binomial Equation
            \begin{itemize}
                \item $p + q = 1$
                \item $p^2 + 2pq + q^2 = 1$
                \item $p^2$ = frequency of homozygous dominant genotype
                \item $q^2$ = frequency of homozygous recessive genotype
                \item $2pq$ = frequency of heterozygous genotype
                \item $p$ = frequency of dominant allele
                \item $q$ = frequency of recessive allele
            \end{itemize}
            \item Conditions
            \begin{enumerate}
                \item No mutations - alleles do not change or changes that do occur are balanced by opposite changes
                \item No gene flow -  no new alleles are added to the gene pool or no alleles are lost from the gene pool due to migration of individuals into or out of the population
                \item Random mating - individuals pair by chance, not according to genotype or phenotype
                \item No genetic drift - population is large enough to prevent random changes in allele frequencies
                \item No selection or selective pressure - no alleles are favored over others
            \end{enumerate}
            \item If $p$ or $q$ is changed in the next generation, then the population is evolving
            \item Identifies factors that cause evolution
            \item If the population is not in Hardy-Weinberg equilibrium, then it is evolving
            \item Conditions are rarely met
            \item Provides starting point for studying mechanics of evolution
        \end{itemize}
        \item Natural Selection
        \begin{itemize}
            \item Adaption of a population to the biotic and abiotic environment
            \item Biotic:\ living factors (predators, prey, parasites, competitors)
            \item Abiotic:\ nonliving factors (temperature, humidity, pH, salinity, sunlight)
            \item Types of Selection
            \begin{enumerate}
                \item Directional Selection:\ favors one extreme phenotype
                \item Stabilizing Selection:\ favors intermediate phenotype
                \item Disruptive (Diversifying) Selection:\ favors two or more extreme phenotypes
                \item Balacing Selection:
                \begin{itemize}
                    \item Maintains genetic diversity
                    \item Balancecd Polymophisms:\ two or more alleles are kept in balance and therefore are maintained in a population over the course of many generations
                    \item Two common ways: heterozygote advantage and frequency-dependent selection (rare individuals have an advantage)
                \end{itemize}
            \end{enumerate}
        \end{itemize}
    \end{itemize}
\end{document}