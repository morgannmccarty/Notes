\documentclass[12pt]{article}

% chktex-file 8

\title{
    General Biology 2: Lecture 3}
\author{Morgan McCarty}
\date{13 September 2023}

\begin{document}
    \maketitle

    \begin{itemize}
        \item History of Life
        \begin{itemize}
            \item Adaptive Radiation
            \begin{itemize}
                \item Post-Mass Extinction
                \item Surviving species quickly diversif
            \end{itemize}
            \item Precambrian Time
            \begin{itemize}
                \item Hadean, Archaean, Proterozoic Eons
                \begin{itemize}
                    \item Little to no atmospheric oxygen
                    \item Lack of ozone shield allowed radiation to bombard Earth
                    \item First cells come into existence in aquatic environments
                    \item Procaryotes (3.5 billion years ago) \\
                    Cyanobacteria left many ancient stromatolite fossils \\
                    Cyanobacteria added first oxygen to the atmosphere \\
                    Evolution of abiotic species
                    \item Eukaryotes (2.5 billion years ago) - (Endosymbiotic Hypothesis: eukaryotes evolved from prokaryotes)
                    \item Multicellularity Arises (1.5 billion years ago)
                    \item Glycolysis as first aerobic process
                    \item Union of bacteria and archaean potentially led to the first Eukaryotic cell (membranes)
                    \item Ediacaran Fossils (end of Proterozoic) 600-540 million years ago \\
                    Multicelluar animals appear including sponges \\
                    Shallow marine mudflat animals, unusual forms, no interal organs, no shell or bones (all invertibrates) \\
                    Possesed collagen (all animals have the collagen protein) \\
                    Ended with a Mass Extinction event \\
                    Cylindrical / segmented fossils from Ediacaran period show signs of animals being more elaborate, but most fossils are not discovered
                \end{itemize}
            \end{itemize}
            \item Phanerozoic Eon
            \begin{itemize}
                \item Paleozoic Era (``Ancient Life'') 540-248 million years ago
                \begin{itemize}
                    \item Cambrian Explosion \\
                    Warm, wet climate, $O_2$, no ice at poles \\
                    All existing phyla appear in the fossil record \\
                    No new animal body plans have developed since the Cambrian Explosion \\
                    Many marine invertibrates with shells \\
                    First vertibrates (520 million years ago)
                    \item High diversity of the Cambrian due to: \\
                    Favorable environment - Oxygen, (Calcium Carbonate for shells) \\
                    Evolution of Hox genes (regulatory genes) \\
                    Predator/prey ``Arms Race'' - shells, reef-building
                    \item Burgess Shale organisms \\
                    British Columbia, Canada \\
                    Rapid burying of animals in mudslide led to rapid fossilization of many species \\
                    Continuous new discoveries e.g.\ massive new species of arthropod (radiodonts)
                    \item Ordovician Period (490-443 million years ago) \\
                    Warm temperatures and atmosphere very moist, lots of $CO_2$ in atmosphere \\
                    Diverse marine invertibrates: trilobites, brachiopods, byozoans, etc. \\
                    Primitive plants and arthopods first invade land \\
                    First invertibrates (fish-like) \\
                    Abrupt climate change (glaciers) led to mass extinction
                    \item Dilurian Period (443-417 million years ago) \\
                    Stable climate, glaciers melted, sea levels rose \\
                    Significant vertibrates (fish), plants, coral reefs \\
                    Large colonization by terrestrial plants (seedless) and arthropods
                    \item Devonian Period (417-354 million years ago) \\
                    ``Age of Fishes'' \\
                    Rapid diversification of fishes \\
                    North is dry, south is wet (oceans) \\
                    Jawed and unjawed fishes gain dominance of cephalopods
                    \item Carboniferous Period (354-290 million years ago) \\
                    Rich coal deposits formed from plant material \\
                    Cooler with land covered by forests and swamps \\
                    Plants and animals further diversified \\
                    Very large plants and trees present \\
                    Flying Insects \\
                    Animals developed in isolation tend to be bigger\\
                    Amphibians prevalent \\
                    Amniotic egg evolved in reptiles (leathery egg shell)\\
                    Amphibians lay eggs in water, reptiles eggs protected by a shell and can be layed on land, provides internal fluid for embryo
                    \item Permian Period (290-248 million years ago) \\
                    Continental drift formed supercontinent Pangaea \\
                    Forest shift to gymnosperms (conifers) \\
                    Amphibians prevalent, but reptiles begin to dominate \\
                    First mammal-like animals appeared //
                    Ended with the largest known mass extinction event (``The Great Dying'') \\  
                \end{itemize}
                \item Mesozoic Era (``Middle Life'') 248-65 million years ago
            \end{itemize}
        \end{itemize}
    \end{itemize}
\end{document}