\documentclass[12pt]{article}

% chktex-file 8

\title{
    General Biology 2: Lecture 5}
\author{Morgan McCarty}
\date{18 September 2023}

\begin{document}
    \maketitle

    \begin{itemize}
        \item Survival of the Fit
        \begin{itemize}
            \item Fitness is the relative reproductive success of an individual
            \item Most-fit individuals in a population capture a disproportionate amount of the resources and are more likely to pass along their genes
        \end{itemize}
        \item Rock Pocket Mice
        \begin{itemize}
            \item Adaptive melanism (darkening of skin) in response to volcanic rock
        \end{itemize}
        \item Evolution in Action - Industrial Melanism
        \begin{itemize}
            \item Prior to industrial revolution, light-colored moths were more common ($90\%$) than dark-colored moths ($10\%$)
            \item After industrial revolution, dark-colored moths were more common ($80\%$) than light-colored moths ($20\%$)
            \item Soot in the atmosphere killed lichens and darkened trees, birds act as a selective agent
            \item Now, light-colored moths are more common again ($80\%$) than dark-colored moths ($20\%$)
        \end{itemize}
        \item Evidence of Evolution
        \begin{itemize}
            \item Comparative anatomy
            \begin{itemize}
                \item Homologous structures:\ similar structures in different species due to common ancestry (not neccessarily similar function)
                \item Analogous structures:\ similar structures in different species due to convergent evolution (similar function, not common ancestry)
                \item Vestigial structures:\ structures that are reduced forms of functional structures (still fully developed) in other organisms (e.g.\ human appendix)
            \end{itemize}
            \item Comparative development
            \begin{itemize}
                \item All vertibrate embryos have a tail and pharyngeal pouches
                \item ``Onatogeny recapitulates phylogeny'' - Ernst Haeckel
            \end{itemize}
            \item Fossil record
            \begin{itemize}
                \item Fossils record the history of life from the past
                \item Document a succession of life forms from simple to more complex
                \item Sometimes the fossil reocrd is complete enough to show descent from an ancestor
            \end{itemize}
            \item Biogeography
            \begin{itemize}
                \item Alfred Russel Wallace
                \item Study of geographical distribution of plants and animals across Earth
                \item Different mixes of plants and animals based on geographical location
                \item Different land massses separated by oceans
                \item e.g.\ Marsupials
            \end{itemize}
            \item Molecular Homologies
            \begin{itemize}
                \item Almost all living organisms use same basic biochemical molecules and utilize the same DNA triplet code (same 20 animo acids in proteins)
            \end{itemize}
            \item Genetic Homologies - Cytochrome C
            \begin{itemize}
                \item Human and Chimpanzees identical
                \item Chickens and turkeys identical; differing from ducks by 1
                \item Humans and chickens differ by 13
            \end{itemize}
        \end{itemize}
        \item Process of Evolution
        \begin{enumerate}
            \item Variations are produced by chance mutations and sexual reproduction
            \item Natural selection selects the ``fittest'' organisms
            \item Natural selection leads to adaptions to a particular environment
            \item Process occurs constantly in all species of life on Earth
        \end{enumerate}
        \begin{itemize}
            \item Natural selection acts on individuals, but evolution occurs in populations
        \end{itemize}
    \end{itemize}
\end{document}