\documentclass[12pt]{article}

% chktex-file 8

\title{
    General Biology 2: Lecture 4}
\author{Morgan McCarty}
\date{14 September 2023}

\begin{document}
    \maketitle

    \begin{itemize}
        \item 6th Mass Extinction - ``Worst state of species die off since loss of dinosaurs''
        \begin{itemize}
            \item Humans:\ ``Global superpredators''
            \item Previous:\ $100\%$ wildlife;\ Now:\ $3\%$ wildlife
            \item $CO_2$:\ Permian: $2-3000$ppm;\ Now: $414$ppm (rising)
            \item Human activities have caused atmospheric $CO_2$ to rise $150\%$ since 1750
        \end{itemize}
        \item Since dawn of human civilization
        \begin{itemize}
            \item Loss of
            \begin{itemize}
                \item $83\%$ of wild mammals
                \item $80\%$ of marine mammals
                \item $50\%$ of plants
                \item $15\%$ of fish
            \end{itemize}
            \item Large cats estimated to be extinct in the wild in 10-15 years
            \item Rhinos, pangolins, giraffes, etc.\ are all close to being extinct
        \end{itemize}
        \item Living Planet Report
        \begin{itemize}
            \item Average $69\%$ decline in monitored population sof mammals, birds, amphibians, reptiles, fish (vertibrates only)
            \item Most significant declines in Americas and Africa
            \item Freshwater biodiversity declining faster than terrestrial or oceanic
            \item Megafauna (large animals) are particularly vulnerable
            \item Plants dying at same rate as mammals, higher than birds
            \item $20\%$ of wild species at risk of extinction this century
            \item Many primates are on the brink of extinction
            \begin{itemize}
                \item Political instability
                \item Hunting
                \item Habitat loss
                \item e.g.\ Graver's gorilla, Aye-ayes, Northern sportive lemur, Pygmy tarsier, Rondo dwarf galago, Slow loris, etc.
            \end{itemize}
        \end{itemize}
        \item Evolution:\ Heritable change in one or more characteristics of a population or specie from one generation to the next
        \begin{itemize}
            \item Micro-evolution:\ single gene in a population
            \item Macro-evolution:\ formation of new species or groups of species through accumulation of micro-evolutionary changes
            \item Species:
            \begin{itemize}
                \item Group of related organisms that share a distinctive form
                \item Among sexually reprodutive species, able to interbreed
            \end{itemize}
            \item Population:\ a localized group of individuals from a species which are more likely to interbreed
        \end{itemize}
        \item History of Evolutionary Thought
        \begin{itemize}
            \item Predarwinian - Theology, Myths, Superstition
            \item Anaximander - Plato, Aristotle:\ \textit{Scala Naturae} (``Scale of Life'')
            \begin{itemize}
                \item Estabilishes man as the dominant/perfect form of Life
                \item Set man above and apart from nature
                \item Incorporated into religious belief that the Earth and its creatures were the result of creation
                \item ``Great Chain of Being''
            \end{itemize}
            \item Creationism:\ A God is the absolute creator of heaven and Earth out of nothing (Christianity, Islam, Judaism)
            \item Spontaneous Generation:\ Life arises from non-living matter
            \begin{itemize}
                \item Sweaty rages with grain prduces mice $\rightarrow$ sweat transformed grain into rice
            \end{itemize}
            \item Taxonomy - 17th Century to mid-18th Century
            \begin{itemize}
                \item Jon Ray - first thorough study of natural world
                \item Carolus Linnaeus - binomial nomenclature, fixity of species (ideal structure and function, species do not change)
                \item Count George Buffon - 44 volume catalog of all known plants and animals, suggested life forms change over time
                \item Erasmus Darwin - Charles Darwin's grandfather, suggested all living things descended from a common ancestor, evidence in developmental patters, artificial selection, and vestigial organs
            \end{itemize}
            \item Evolutionary Thought - late 18th Century
            \begin{itemize}
                \item Curvier 
                \begin{itemize}
                    \item First to use comparative anatomy to develope a system of classification
                    \item Founded Paleoentology - fossils
                    \item Proposed Catastrohpism - after catastrophies new species emerge
                \end{itemize}
                \item Lamarck
                \begin{itemize}
                    \item Propose evolution and life diversity with environmental adaption
                    \item More complex organisms are descended from less complex organisms
                    \item Use and disuse - body parts used extensively become larger and stronger, while those not used deteriorate
                    \item Inheritance of acquired characteristics - modifications acquired during lifetime can be passed on to offspring
                    \item Giraffes - long necks from stretching to reach leaves
                    \item Generally rejected by scientists, however some recent evidence suggests that some acquired characteristics can be inherited - traumatic experiences led to epigenetic changes
                \end{itemize}
                \item Charles Lyell - ``Principals of Geology''
                \begin{itemize}
                    \item Earth is subject to slow but continuous cycle of erosion and uplift
                    \item Proposed Uniformitarianism
                    \item Important writings that influenced Darwin
                \end{itemize}
                \item Charles Darwin
                \begin{itemize}
                    \item HMS Beagle - 1831 at age 22
                    \item Naturalist
                    \item 5 years seasick
                \end{itemize}
                \item Darwin's Theory of Evolution
                \begin{itemize}
                    \item Biogeographical observations
                    \item Study o fthe geographic distribution of life forms on Earth
                    \item Saw similar species in similar habitats
                    \item Reasoned related species could be modified according to the environment
                \end{itemize}
            \end{itemize}
            \item Galapagos Islands
            \begin{itemize}
                \item Tortoises
                \begin{itemize}
                    \item Darwin observed tortoise neck length varied from island to island
                    \item Proposed that speciation on islands correlated with a difference in vegetation
                \end{itemize}
                \item Finches
                \begin{itemize}
                    \item Darwin observed many difference species of finches (13) on various islands
                    \item Speculated they could have desceded from a single pair of mainland finches
                \end{itemize}
            \end{itemize}
            \item \textit{On the Origin of Species by Means of Natural Selection} - Darwin
            \begin{itemize}
                \item Rev. Thomas Malthus - ``Essay on the Principle of Population''
                \begin{itemize}
                    \item Struggle for existance:\ generations have same reproductive potential as previous generations, but resources are limited
                    \item Reproductive potential is greater than environment can support
                    \item Death, disease, and famine are inevitable if population is to have stability
                \end{itemize}
                \item Darwin's Explanatory Model of Evolution by Natural Selection
                \begin{itemize}
                    \item Observations
                    \begin{enumerate}
                        \item Organisms have great potential fetility, which permits exponential growth of population
                        \item Natual populations do not normally increase exponentially, but remain fairly consistent in size
                        \item Natural resources are limited
                        \item Variation occurs among organisms within populations
                        \item Variation is heritable
                    \end{enumerate}
                    \item Inferences
                    \begin{enumerate}
                        \item A struggle for existence occurs among organisms in a population
                        \item Varying organisms show differential survival and reproduction, favoring advantageous traits (natural selection)
                        \item Natural selection, acting over many generations, gradually produces new adaptions and new species
                    \end{enumerate}
                \end{itemize}
            \end{itemize}
        \end{itemize}
    \end{itemize}
\end{document}