\documentclass[12pt]{article}

\usepackage{hyperref}
\usepackage[margin=0.5in]{geometry}
\usepackage[fleqn]{amsmath}
\usepackage{amsfonts}
\usepackage{amsthm}
\usepackage{amssymb}
% \usepackage{tikz}

% chktex-file 8

\title{
    General Biology 2: Lecture 1}
\author{Morgan McCarty}
\date{07 September 2023}

\begin{document}
    \maketitle

    \begin{itemize}
        \item \underline{Biology}: the \underline{scientific} study of \underline{living organisms} and how they have evolved
        \item \underline{Evolutionary Theory} $\rightarrow$ unity, diversity
        \begin{quote}
            ``Nothing in Biology makes sense except in the light of evolution''
        \end{quote}
        \begin{quote}
            ``Descent with Modification''
        \end{quote}
        \item Characteristics of (all) Living Organisms (Unity)
        \begin{itemize}
            \item Cells and Organization
            \begin{itemize}
                \item Organisms maintain internal order
                \item Chemical uniformness
                \item Hierarchy of Organization
                \begin{enumerate}
                    \item Cells
                    \item Tissues
                    \item Organs
                    \item Indiviuals
                    \item Species
                \end{enumerate}
                \item Emergent properties
                \begin{itemize}
                    \item Sum of the parts creates unique properties
                \end{itemize}
            \end{itemize}
            \item Energy Use and Metabolism
            \begin{itemize}
                \item Energy required to maintain order
                \item Energy used via metabolism
                \item Photosynthesis; Cellular respiration; ATP (glycolosis)
            \end{itemize}
            \item Response to Environmental Change
            \begin{itemize}
                \item Organisms react to stimuli
                \item Adaptations/behaviors promote survival
            \end{itemize}
            \item Regulation and Homeostasis
            \begin{itemize}
                \item Organisms regulate cells and bodies
                \item Maintain relatively stable internal conditions \\
                    (\underline{Disease}: inability to maintain homeostatis)
            \end{itemize}
            \item Growth, Development, and Reproduction
            \begin{itemize}
                \item Growth produces more and/or larger cells
                \item Development produces organisms with defined sets of characteristics (DNA)
                \item Reproduction sustains species over generations
                \item Genetic material causes offspring to have traits like their parents
            \end{itemize}
            \item \textbf{Biological Evolutions}
            \begin{itemize}
                \item Populations of organisms change over generations
                \item Evolution results in traits that promote survival and reproductive success
            \end{itemize}
        \end{itemize}
        \item Core Concepts for Biological Literacy
        \begin{itemize}
            \item Evolution
            \item Structure and Function
            \item Information Flow, Exchange, and Storage
            \item Pathways and Transformations of Energy and Matter
            \item Systems
        \end{itemize}
        \item Structure Determines Function
        \begin{itemize}
            \item Biological structures come about as a species adapts to its environment
            \item Example: Human Hand
            \begin{itemize}
                \item Graps things (fine control)
                \item Grab Objects (power)
                \item Structure to function relationship:
                \begin{itemize}
                    \item Opposible thumb $rightarrow$ touching fingers to base of hand (power grip)
                \end{itemize}
            \end{itemize}
            \item Example: Bat Flight
            \begin{itemize}
                \item Bat fingers have a thing flap of skin that allows lift
            \end{itemize}
        \end{itemize}
        \item All species (past and present) are related by an evolutionary history
        \begin{itemize}
            \item Vertical descent with mutation
            \item Horizontal Gener Transfer (non-offspring)
            \begin{itemize}
                \item Different species transfering DNA between each other (e.g.\ bacteria)
                \item Process that likely led to the origins of procayotes (cell nucleus)
            \end{itemize}
        \end{itemize}
        \item Biological Evolution
        \begin{itemize}
            \item \underline{Adaptation}: any modification that makes an organism better suited to its way of life
            \item \underline{``Descent with Modification''}
            \begin{itemize}
                \item Populations of organisms change over generations
                \item Evolution results in adaptations
                \item Better adapted organisms tend to survive and produce more offspring
            \end{itemize}
        \end{itemize}
        \item How did Life Begin?
        \begin{enumerate}
            \item $13-17$ bya: Big Bang
            \item $4.6$ bya: Solar System forms
            \item $4.55$ bya: Earth forms
            \item $4$ bya: Earth cools enough for outer layers and oceans to form
            \item $3.5-4$ bya: Life emerges
        \end{enumerate}
        \begin{itemize}
            \item Life requires interplay between RNA, DNA, and proteins
            \item Living cells come from prexisting cells
            \item Key steps:
            \begin{itemize}
                \item Nucleotides and amino acids produced prior to life
                \item Nucleotides and amino acides become polymerized to form DNA, RNA, proteins
                \item Polymers become enclosed in membranes
                \item Polymers enclosed in membrances exhibit ceullular properties
            \end{itemize}
            \item Primitize Earth
            \begin{itemize}
                \item Reducing Atmosphere Hypothesis
                \begin{itemize}
                    \item Atmosphere primarily consisted of $H_2O$ vapor, $N_2$, $CO_2$, small amounts of $H_2$ and $CO$
                    \item Little free oxygen
                    \item Primordial soup (nutrient rich)
                    \item Spontaneous formation of organic molecules
                    \item Monomers evolved and joined to form polymers
                    \item \underline{Abiotic Systhesis} (pre-biotic)
                    \item Theory given in $1920$s by Oparin and Haldane
                    \item First Biomolecules
                    \item Miller and Urey ($1953$): Apparatus Experiment
                    \item Showed that biochemicals could be produced from simpe non-biological sources
                    \item Primitize atmospheric gases
                    \item Strong energy sources
                    \item Yielded $HCN$, $CH_2O$, glycene, sugars, amino acids, N-bases
                    \item More recent:
                    \item Natural environment: $CO$, $CO_2$, $N_2$, $H_2O$
                    \item Organics can be made under a variety of conditions
                    \item Reducing (high pH) or neutral environments, both produce molecules
                \end{itemize}
                \item Alternative Mechanisms?: Extraterrestrial Hypothesis                        
                \begin{itemize}
                    \item Organic carbon from asteroids and comets impacting prebiotic soup
                    \item Controversy: destroy by intense heat of impact
                \end{itemize}
                \item Deep Sea Vent Hypothesis
                \begin{itemize}
                    \item Key organics arose at deep-sea vents
                    \item Superheated water ($300\deg$F) rich with $H_2S$ and metal ions
                \end{itemize}
            \end{itemize}
            \item Origin of the 1st Cell
            \begin{itemize}
                \item Clay Hypothesis
                \begin{itemize}
                    \item Simple organics polymerize on solid surfaces (clay, mud, inorganic crystals) into complex organics
                \end{itemize}
                \item Cell-Like Structures - Protobionts
                \begin{itemize}
                    \item Boundary (i.e.\ membrane)
                    \item Polymers inside contain information
                    \item Self-replicattion
                \end{itemize}
                \item Chemical Solution - RNA World
                \begin{itemize}
                    \item RNA in Protobionts: \\
                        Can store information, self-replicate, enzymatic functions (ribozymes)
                \end{itemize}
            \end{itemize}
            \item Advantages of DNA/RNA/Protein World
            \begin{itemize}
                \item Information Storage
                \begin{itemize}
                    \item DNA would have relieved RNA of information storage role and allowed RNA to perform other functions
                \end{itemize}
                \item Metabolism and Other Functions
                \begin{itemize}
                    \item Proteins have a greater catalytic potential and efficiency (than RNA)
                    \item Proteins can perform other functions (e.g.\ structural)
                \end{itemize}
            \end{itemize}
            \item Fossil Record
            \begin{itemize}
                \item \underline{Fossil}: remains and traces of past life or any other direct evidence of past life
                \item \underline{Paleontology}: study of fossils and the fossil record
                \item Most fossils are traces of organisms embedded in sediment
                \item Older fossils in lower lays $\rightarrow$ series of fossil strata (layers)
            \end{itemize}
        \end{itemize}
    \end{itemize}
\end{document}