\documentclass[12pt]{article}

\usepackage{hyperref}
\usepackage[margin=0.5in]{geometry}
\usepackage[fleqn]{amsmath}
\usepackage{amsfonts}
\usepackage{amsthm}
\usepackage{amssymb}
% \usepackage{tikz}

% chktex-file 8

\title{
    General Biology 2: Lecture 1}
\author{Morgan McCarty}
\date{11 September 2023}

\begin{document}
    \maketitle

    \begin{itemize}
        \item \underline{Fossil Record}
        \begin{itemize}
            \item Other fossils also exist, e.g.\ amber, ice, etc.
            \item Relative Dating
            \begin{itemize}
                \item Law of Superposition: older layers are deeper
                \item Fossils found below other fossils are the older fossils
            \end{itemize}
            \item Index Fossils
            \begin{itemize}
                \item Fossils that are:
                \begin{enumerate}
                    \item Widely distributed
                    \item Easy to recognize
                    \item Existed for a long period of time
                \end{enumerate}
                \item They can be used to relatively date other fossils (as a uniform age exists for them)
                \item E.g.
                \begin{itemize}
                    \item ``Ammonites'', a type of mollusk from the Mesozoic era
                    \item ``Trilobites'', a type of arthropod from the Paleozoic era
                    \item and many others
                \end{itemize}
            \end{itemize}
            \item Factors that Affect the Fossil Record
            \begin{itemize}
                \item Anatomy: hard parts fossilize more easily (e.g.\ bones, teeth, shells, etc.) and are more likely to be preserved
                \item Size: fossil remains of larger organisms are more likely to be found
                \item Number: species that exist in greater numbers over wider areas are more likely to be found
                \item Environment: inland species are less likely to be found than marine species due to ease of sedimentation on the ocean floor (species that lived on the edge of the ocean are more likely to be found than inland as well)
                \item Time: species that lived more recently or for a longer period of time are easier to find
                \item Geological Processes: certain organisms are more likely to be destroyed by geological processes (e.g.\ erosion, volcanoes, etc.)
                \item Paleontology: certain types of fossils may be more interesting to paleontologists, and thus more likely to be found (e.g.\ dinosaurs)
            \end{itemize}
            \item Absolute Dating
            \begin{itemize}
                \item Radiometric Dating (Half-Life)
                \begin{itemize}
                    \item Half-life: the time it takes for half of the atoms in a sample to decay to a daughter isotope ($C_{14} \rightarrow N_{14}$ in $5,730$ years)
                    \item Unaffected by temperature, light, pressure, etc.
                    \item All radioactive isotopes have a dependable half-life (from seconds to billions of years)
                \end{itemize}
            \end{itemize}
            \item Geologic Time Scale
            \begin{quote}
                ``The history of life over time''
            \end{quote}
            \begin{itemize}
                \item Biologic history based off of fossil evidence
                \item Changes observed in organisms
                \item Results of genetic changes, environmental changes, etc.
                \item Patterns consistent with:
                \begin{itemize}
                    \item Climate/temp, atmospheric composition, landmassses (continental drift - $10$ cm/year), floods/glaciation, volcanic eruptions, meteorite impacts, etc.
                \end{itemize}
                \item Eras
                \begin{itemize}
                    \item Paleozoic, Mesozoic, Cenozoic (key focused eras in Biology)
                \end{itemize}
                \item \underline{Cambrian Period}
                \begin{itemize}
                    \item Sudden increase of diversity of many animal phyla (``Cambrian Explosion'')
                \end{itemize}
                \item \underline{Permian-Triassic Extinction}
                \begin{quote}
                    ``The Great Dying''
                \end{quote}
                \begin{itemize}
                    \item $81\%$ of marine species went extint (commonly quoted as $90+\%$, perhaps incorrectly)
                \end{itemize}
                \item \underline{Cretaceous-Paleogene Extinction}
                \begin{quote}
                    ``K-T Extinction'' or ``K-Pg Extinction''
                \end{quote}
                \begin{itemize}
                    \item $75\%$ of plant and animal species went extinct (e.g.\ the non-avian dinosaurs)
                \end{itemize}
                \item \underline{Holocene Extinction}
                \begin{itemize}
                    \item Current mass extinction event
                    \item Human induced
                \end{itemize}
            \end{itemize}
        \end{itemize}
    \end{itemize}
\end{document}