\documentclass[12pt]{article}

\usepackage{hyperref}
\usepackage[margin=0.5in]{geometry}
\usepackage[fleqn]{amsmath}
\usepackage{amsfonts}
\usepackage{amsthm}
\usepackage{amssymb}
% \usepackage{tikz}

% chktex-file 8

\title{
    Genetics and Molecular Biology: Lecture 5}
\author{Morgan McCarty}
\date{12 September 2024}

\begin{document}
    \maketitle
    \begin{itemize}
        \item DNA Replication Technology
        \begin{itemize}
            \item Polymerase Chain Reaction (PCR) - 1983
            \begin{itemize}
                \item In vitro (test tube) method for making copies of DNA sequences using DNA polymerase
                \item USes: molecular cloning, diagnostics, forensics, DNA sequencing, gene expression analysis, etc.
                \item E.g. Covid testing
                \item PCR amplifies specific `target' DNA sequences
            \end{itemize}
            \item PCR Primers Initiate Replication
            \begin{itemize}
                \item Short DNA sequences (oligonucleotides) produced synthetically and added to PCR Reaction
                \item Fufill same role as RNA primase (but are DNA)
                \item Primers can be designed to target regions of desire (region amplifed between primers)
            \end{itemize}
            \item PCR Components
            \begin{itemize}
                \item Template (source) DNA
                \item DNA nucleotides (dNTPS) e.g. any nucleotide triphosphates
                \item oligonucleotides primers
                \item Buffers/salts - magnesium buffers triphosphate charge: need proper envrionment for DNA polymerase to work
                \item DNA polymerase
            \end{itemize}
            \item PCR cycle
            \begin{enumerate}
                \item Denature - $95\deg$ C: heat to separate DNA strands
                \item Anneal (prime) - $50-60\deg$ C: cool to allow primers to bind to template DNA
                \item Extend - $72\deg$ C: heat to allow DNA polymerase to synthesize new DNA strand
                \item Repeat
            \end{enumerate}
            \item Heat stable polymerase greatly improves PCR
            \begin{itemize}
                \item Thermus aquaticus (Taq): bacteria that lives in Yellowstone hot springs has DNA polymerase that will not denature during heating of DNA to separate strands
            \end{itemize}
            \item The product of one cycle being used as a template in next cycle leads to an exponential increase in DNA - only works if the product of one primer contains the binding site for the other primer
            \item Primer Sequence influences annealing temperature
            \begin{itemize}
                \item GC base pairs are stronger than AT pairs, so they can anneal at higher temps
                \item Larger sequences anneal at higher temps
            \end{itemize}
        \end{itemize}
        \item DNA Sequencing
        \begin{itemize}
            \item Reasosn to know a DNA molecule's nucleotide sequences
            \begin{itemize}
                \item Find and interpret genes
                \item Asses evoluation connections
                \item Discover and diagnose genetic disorders
            \end{itemize}
            \item Sanger Sequencing - 1977
            \begin{itemize}
                \item Similar to PCR, but only one primer is used and no chain reaction occurs
                \item Dideoxynucleotides (ddNTPs) included to prematurely teminate strand synthesis (they do not have a 3' OH group)
                \item Flourescently labeled ddNTPS + unlabled dNTPS + primer and polymerase + DNA template are added to sequence
                \item Goal to creaate nucleotide stands of each length then sort by size to determine nucelotide at each location
                \item Gel Electrophoresis
                \begin{itemize}
                    \item Uses electricity to separate DNA molecules by size (number of nucelotides)
                    \item Larger molecules move through the gel slower than small moleculear and get stuck sooner
                    \item Add a DNA sample to electric field and it slowly moves toward the postive end (nucleic acids are negatively charged)
                \end{itemize}
                \item Capillary Electrophoresis
                \item Sanger sequencing can only sequence 100s to 1000s of base pairs at a time (a single read)
                \item Human genome has 3.2 billion base pairs (shotgun sequencing used to sequnce many disjointed small portions)
            \end{itemize}
            \item Contigs and Coverage (Genome Sequencing Assembly)
            \begin{itemize}
                \item Contig: continuous segment of sequence created through overalapping reads (sequences)
                \item Coverage: number of copies of genome sequnced, more coverage is needed for new genome assembly
            \end{itemize}
        \end{itemize}
    \end{itemize}
\end{document}