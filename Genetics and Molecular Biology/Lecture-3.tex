\documentclass[12pt]{article}

\usepackage{hyperref}
\usepackage[margin=0.5in]{geometry}
\usepackage[fleqn]{amsmath}
\usepackage{amsfonts}
\usepackage{amsthm}
\usepackage{amssymb}
% \usepackage{tikz}

% chktex-file 8

\title{
    Genetics and Molecular Biology: Lecture 3}
\author{Morgan McCarty}
\date{09 September 2024}

\begin{document}
    \maketitle

    \begin{itemize}
        \item Genome and Chromosome Structure
        \begin{itemize}
            \item Genomes consist of one or more chromosomes - a large continuous DNA molecule
            \item Classes of Cells
            \begin{itemize}
                \item Prokaryotic cells ``before nucleus'':
                \begin{enumerate}
                    \item Archaea, Eubacteria (domains)
                    \item 1-5 $\mu$m in diameter
                    \item DNA stored in cytoplasm
                    \item Only single-celled organisms
                \end{enumerate}
                \item Eukaryotic cells ``true nucleus'':
                \begin{enumerate}
                    \item Eukarya (domain)
                    \item DNA stored in nucleus (membrane-bound)
                    \item Organelles with specialized tasks
                    \item 10-100 $\mu$m in diameter
                    \item Single-celled and multicellular organisms
                \end{enumerate}
            \end{itemize}
            \item Genome Architecture
            \begin{itemize}
                \item Prokaryotes:
                \begin{enumerate}
                    \item Very diverse
                    \item One or more chromosomes (linear or circular)
                    \item Can also have circular pplasmids not needed to survive: smaller than chromosomes, gained or lost due to environmental conditions
                \end{enumerate}
                \item Eukaryotes:
                \begin{enumerate}
                    \item Multiple linear chromosomes
                    \item Chromosomes do not loop
                \end{enumerate}
            \end{itemize}
            \item Genome Size: \textit{E. coli}: 4639 kilobases (kb) (0.006 ft); Human: 3200000 kb (6 ft)
            \item Chromosome rearrangement causes chromosome number to vary between species
        \end{itemize}
        \item Eukaryotic Chromosomes
        \begin{itemize}
            \item Made of Chromatin - DNA wrapped around histone proteins
            \item Nucleosome - DNA strans wrap as unit around histone "bead on a string"
            \begin{enumerate}
                \item 8 histones per nucleosome
                \item Arrange into higher order packing patters: 30nm filament from nucleosomes $/rightarrow$ extended form of chromosome $/rightarrow$ condensed section of chromosome $/rightarrow$ mitotic chromosome
            \end{enumerate}
            \item Cohesin proteins help pack DNA: bring sister chromatids together during mitosis, repair DNA
        \end{itemize}
        \item Chromosome Condensation
        \begin{itemize}
            \item More tightly packed DNA: genes less likely to be read to make proteins
            \item Euchromatin: loosely packed, DNA actively being used to produce proteins
            \item Heterochromatin: tightly packed, DNA not being read to produce proteins
        \end{itemize}
        \item Covalent Modifications to Histine Proteins and DNA Changes DNA Packing Density
        \begin{itemize}
            \item Acetylation of histone proetins loosens DNA packing (acetyl groups are polar) - acetylated histones are more likely to be Euchromatin
            \item Methylation of histone proteins and DNA tightens DNA packing (methyl groups are nonpolar)
        \end{itemize}
        \item RNA Structure
        \begin{itemize}
            \item RNA vs. DNA
            \begin{itemize}
                \item RNA is single-stranded (usually)
                \item RNA has uracil instead of thymine
                \item RNA uses ribose instead of deoxyribose (sugar backbone)
                \item RNA can also make base pairs either on hte same RNA molecule, on different RNA molecules, or with DNA (with a single strand)
                \item RNA folds into 3D strcutures based on base pairing
            \end{itemize}
        \end{itemize}
        \item DNA Replication
        \begin{itemize}
            \item DNA: Double Helix of Anti-Parallel Strands
            \begin{itemize}
                \item One strand: 5' to 3'
                \item Other strand: 3' to 5'
                \item Sugar phosphate backbones
                \item Base pairs: A $=$ T, G $\equiv$ C
            \end{itemize}
            \item Cell Division
            \begin{itemize}
                \item Prior: DNA must be replicated to make two identical copies for daughter cells
                \item Chromosome duplication and Condensation
                \item Separation of sister chromatids into two chromosomes
            \end{itemize}
            \item Complementary strands allow for sinple replication of DNA
            \item Definitions:
            \begin{itemize}
                \item \underline{Origin of Replication}: specific sites (sequences) where replication starts
                \item \underline{Replication Bubble}: expanded area of replicated DNA
                \item \underline{Replication Fork}: site of active replication, two per Bubble
            \end{itemize}
        \end{itemize}
    \end{itemize}
\end{document}