\documentclass[12pt]{article}

\usepackage{hyperref}
\usepackage[margin=0.5in]{geometry}
\usepackage[fleqn]{amsmath}
\usepackage{amsfonts}
\usepackage{amsthm}
\usepackage{amssymb}
% \usepackage{tikz}

% chktex-file 8

\title{
    Genetics and Molecular Biology: Lecture 2}
\author{Morgan McCarty}
\date{05 September 2024}

\begin{document}
    \maketitle

    \begin{itemize}
        \item DNA Function
        \begin{itemize}
            \item What does DNA actually do?
            \item Central Dogma:
            \begin{center}
                DNA $\rightarrow_{\text{transcription}}$ RNA $\rightarrow_\text{folding}$ Amino Acid Chain $\rightarrow_{\text{translation}}$ Protein \\
                DNA $\rightarrow_{\text{replication}}$ DNA
            \end{center}
            \item Most Ceullar Processes Depend on Proteins
            \begin{itemize}
                \item Catalysis
                \item Movement
                \item Structure
                \item Communication
                \item Transport
            \end{itemize}
            \item Proteins are linear chains of amino acids which fold to make complex shapes capable of doing a specific task in the cell
            \item Order of amino acids in proteins determins its shape and abilities
            \item Sickle Cell Disease: A change in one amino acid causes clumping due to changes in folding
        \end{itemize}
        \item DNA Structure
        \begin{itemize}
            \item DNA: two strands of nucleotides, covalent bonds link within-strands, hydrogen bonds link between strands
            \item Four difference nucleotides with four bases
            \begin{itemize}
                \item Thymine, Cytosine (Pyrimidines)
                \item Adeine, Guanine (Purines)
            \end{itemize} 
            \item Nucleotides are linked in strands
            \item Nucleotides and strands are asymmetric
            \begin{itemize}
                \item 5' carbon atom o fthe sugar is where the phosphate group is attached
                \item 3' carbon atom of the sugar has a hydroxyl group
            \end{itemize}
            \item Stands are made by attaching 5' phosphate to 3' hydroxyl of next nucleotide
            \item Phosphodiester linkage
            \item Information in DNA is stored in order of nitrogenous bases
        \end{itemize}
        \item Erwin Chargaff - 1952
        \begin{itemize}
            \item In organisms DNA:
            \begin{center}
                Amount of Adenine = Amount of Thymine \\
                Amount of Cytosine = Amount of Guanine \\
                Amount of Purines = Amount of Pyrimidines
            \end{center}
        \end{itemize}
        \item Rosalind Franklin - 1953
        \begin{itemize}
            \item X-ray crystallography of DNA
            \begin{center}
                crystals $\rightarrow_{\text{x-rays}}$ diffraction pattern $\rightarrow_{\text{phases}}$ electron density map $\rightarrow_{\text{fitting}}$ structure
            \end{center}
        \end{itemize}
        \item James Watson and Francis Crick - 1953
        \begin{itemize}
            \item Double helix model of DNA
        \end{itemize}
        \item Base Pairing
        \begin{itemize}
            \item (Purine) : (Pyrimidine)
            \item Because of base pairing, DNA strands complement each other
        \end{itemize}
        \item Anti-Parallel Strands
        \begin{itemize}
            \item Key Features:
            \begin{itemize}
                \item Anti-parallel strands: one strand runs 5' to 3', the other runs 3' to 5'
                \item Sugar phosphate backbones
                \item Bases glue strands together with H-bonds
                \item A $=$ T, G $\equiv$ C
            \end{itemize}
        \end{itemize}
        \item DNA can form different helix structures under different conditions
        \begin{itemize}
            \item Common form is B-DNA
            \item A-DNA and Z-DNA are also possible
        \end{itemize}
        \item DNA Helices are mostly right-handed (B-DNA and A-DNA, Z-DNA is left-handed)
        \item Base Pairing angles have major and minor grooves
        \begin{itemize}
            \item Major groove - more access to nitrogen bases
            \item Minor groove - more access to sugar and phosphate
        \end{itemize}
        \item Helices can be overwound or underwound causing supercoiling
        \begin{itemize}
            \item Underwinding creates negative supercoils (right handed rossing) \\
            Can assist in strand separation
            \item Overwinding creates positive supercoils (left handed crossing)
        \end{itemize}
        \item Topoisomerase Enzymes Change Winding/Coiling
        \begin{itemize}
            \item Type I: Separates one strand, twists, reconnects
            \item Type II: Recognizes the entanglement and makes reverseable covalent attachment to the opposite strands, breaking a strand and forming a gate. Passes the other strand through the gate, then reseals the break \\
            Type II topoisomerase enzymes can decatenate circular DNA (circular chromosomes become catenated - interlocked - during replication)
        \end{itemize}
    \end{itemize}
\end{document}