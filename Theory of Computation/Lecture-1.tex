\documentclass[12pt]{article}

\usepackage{hyperref}
\usepackage[margin=0.5in]{geometry}
\usepackage[fleqn]{amsmath}
\usepackage{amsfonts}
\usepackage{amsthm}
\usepackage{amssymb}
% \usepackage{tikz}

% chktex-file 8

\title{
    Theory of Computation: Lecture 1}
\author{Morgan McCarty}
\date{07 September 2023}

\begin{document}
    \maketitle

    \begin{enumerate}
        \item \underline{Theory of Computation}
        \begin{itemize}
            \item What is computation?
            \begin{itemize}
                \item Math-checking if no.\ prime
                \item Verifying logic
                \item Navigational routing
                \item Comparing strings $\rightarrow$ looking up, sorting, etc.\ 
            \end{itemize}
            \item All computation relative to computer? Can we do ths w/o the computer?
            \item What is a computer? Modeling a computer
            \begin{itemize}
                \item Supported operations
                \item Determinsitic (usually)
                \item Output
                \item Input
            \end{itemize}
            \item Are there fundamental limits on what is computable? \\
                \underline{Yes}: Halting problem
            \begin{itemize}
                \item Does a program actually halt?
            \end{itemize}
            \item Areas of the course: (in order of decreasing complexity)
            \begin{enumerate}
                \item Computability: what can be computed given enough time and space
                \item Complexity: how fast/efficiently can we solve a problem
                \item Automata: what problems can we solve given very limited space (constant)
            \end{enumerate}
            \item Why does this matter?
            \begin{itemize}
                \item Checking correctness of a program
                \item Knowinig what functions can be computed quickly and which cannot (security)
            \end{itemize}
            \item Goals of the course:
            \begin{enumerate}
                \item Understand notions of computability
                \item Understand limitations of computability
                \item What can be doen with weaker forms of computability
                \item Computational relation to formal languages
            \end{enumerate}
        \end{itemize}
        \item Mathematical Review
        \begin{itemize}
            \item Sets:
            \begin{itemize}
                \item Unordered group of elements (finite or infinite)
                \item E.g.
                \begin{itemize}
                    \item $\mathbb{N} = \{1, 2, \ldots\}$
                    \item $\mathbb{N} \cup {0}$
                    \item $\mathbb{Z} = \{\ldots, -2, -1, 0, 1, 2, \ldots\}$
                    \item $\emptyset = \{\}$
                \end{itemize}
                \item Set Operations:
                \begin{itemize}
                    \item $X \cup Y = \{x \mid x \in X \lor x \in Y\}$
                    \item $X \cap Y = \{x \mid x \in X \land x \in Y\}$
                    \item $\bar{X}$: negaton of set relative to universe
                    \item $X \setminus Y = X - Y = \{x \mid x \in X \land x \notin Y\} = X \cap \bar{Y}$
                \end{itemize}
            \end{itemize}
            \item Logic:
            \begin{itemize}
                \item $\land$: and
                \item $\lor$: or
                \item $\implies$: implication
                \item $\alpha \implies \beta$ \\
                ``if $\alpha$ is truse, then $\beta$ is true'' \\
                ``if Corina gets fed, then Ariel sleeps in''
                \item Negation of $\alpha \implies \beta$ is $\alpha \land \bar{\beta} \equiv \bar{\alpha} \lor \beta$
                \item Satisfiable: the formula has a set of boolean assignments so that the entire thing evaluates to true
            \end{itemize}
            \item Proof Techniques:
            \begin{itemize}
                \item Proof by induction
                \begin{itemize}
                    \item Induction Prove: $\forall n \in \mathbb{N}_0, P(n)$
                    \item $P(n)$: predicate (boolean statement about $n$)
                    \item \underline{Base Case}: $P(0)$ (or $P(1)$)
                    \item \underline{Inductive Step}: Assume $P(n)$ is true, show $P(n+1)$ is true
                    \item E.g.
                    \begin{itemize}
                        \item Show $5+10+15+\cdots+5n = \frac{5n(n+1)}{2} \forall n \in \mathbb{N}$
                        \item Base Case: $n=1$ \\
                        $5 = \frac{5(1)(2)}{2} = 5$
                        \item Inductive Step: We know for $n = k$ \\
                        $5+10+15+\cdots+5k = \frac{5k(k+1)}{2}$
                        \item Show for $n = k+1$: \\
                        $5 + 10 + 15 + \cdots + 5k + 5(k+1) = \frac{5k(k+1)}{2} + 5(k+1)$ \\
                        $= \frac{5k(k+1) + 10(k+1)}{2} = \frac{5(k+1)(k+2)}{2}$
                    \end{itemize}
                \end{itemize}
                \item Proof by contradiction
                \begin{itemize}
                    \item Assume the opposite of proof statement and show that it leads to a contradiction of a known fact
                    \item E.g.
                    \begin{itemize}
                        \item Show $\sqrt{2}$ is irrational (cannot be written as $\frac{p}{q}$ where $p, q \in \mathbb{Z}$ and $q \neq 0$)
                        \item Assume $\sqrt{2}$ is rational
                        \item $\sqrt{2} = \frac{p}{q}$ where $p, q \in \mathbb{Z}$ and $q \neq 0$
                        \item $p$ and $q$ are relatively prime (no common factors)
                        \item $q\sqrt{2} = p$
                        \item $2q^2 = p^2$
                        \item $p^2$ is even $\implies p$ is even
                        \item $p = 2k$ for some $k \in \mathbb{Z}$
                        \item $2q^2 = {(2k)}^2 = 4k^2$
                        \item $q^2 = 2k^2$
                        \item $q^2$ is even $\implies q$ is even
                        \item $p$ and $q$ are both even, but this contradicts the fact that $p$ and $q$ are relatively prime
                        \item $\therefore \sqrt{2}$ is irrational
                    \end{itemize}
                \end{itemize}
                \item Proof by construction
                \item Proof by contrapositive
                \item Proof by reduction
            \end{itemize}
            \item \underline{Alphabet}
            \begin{itemize}
                \item $\Sigma$: finite set of symbols (``letters'', ``elements'')
                \item E.g.
                \begin{itemize}
                    \item $\Sigma = \{0, 1\}$, $\Sigma = \{a, b, c\}$
                \end{itemize}
                \item A string over $\Sigma$ ($w \in \Sigma$) is a finite sequence of symbols from $\Sigma$. $\Sigma^*$ is the set of all strings over $\Sigma$.
                \item E.g.
                \begin{itemize}
                    \item $w = 010101$, $w = 101010$, or $w = 0000$
                    \item $w =$ aabab, $w =$ ababab, or $w =$ aaa
                    \item $\epsilon$ is the empty string
                \end{itemize}
            \end{itemize}
            \item \underline{Language}
            \begin{itemize}
                \item A language over $\Sigma$ is a set of strings over $\Sigma$
                \item E.g.
                \begin{itemize}
                    \item $L = \{a, ab, aa, bb, \ldots\}$ is a language over $\Sigma = \{a, b\}$
                \end{itemize}
                \item Language is a subset from $\Sigma^*$
                \item How to decide what's in a language?
                \begin{itemize}
                    \item Total list
                    \item Can we do better?
                \end{itemize}
                \item \underline{Machine} to decide the language
                \item $x \in \Sigma^* \rightarrow \boxed{M} \rightarrow Y$ or $N$
                \item Often define an $L$ by the describetion of the $M$ \\
                $M$ \underline{accepts} some strings and \underline{rejects} others \\
                $M$ defines a language $L(M) = \{x|M \text{ accepts } x\}$
            \end{itemize}
            \item \underline{Strings}
            \begin{itemize}
                \item Concatenation: $x \cdot y$ \\
                abc $\cdot$ aab $= $ abcaab
                \item Empty string: $\epsilon$
                a $\cdot \epsilon = \epsilon \cdot a = a$
                \item Length: $|a|$ is the number of elements in a String $a$ \\
                $|$aab$| = 3$, $|\epsilon| = 0$
                \item Prefix: $x, y \in \Sigma^*$, $x$ is a prefix of $y$ if $\exists z \in \Sigma^*$ such that $x \cdot z = y$ \\
                If $x = y$, then $z = \epsilon \iff x$ is a prefix of $y$ and $y$ is a prefix of $x$ \\
                Something is always a prefix of itself
                \item Suffix: $x$ is a suffix of $y$ if $\exists z \in \Sigma^*$ such that $z \cdot x = y$ \\
                If $x = y$, then $z = \epsilon \iff x$ is a suffix of $y$ and $y$ is a suffix of $x$ \\
                Something is always a suffix of itself
                \item Substring: $x$ is a substring of $y$ if $\exists z_1, z_2 \in \Sigma^*$ such that $z_1 \cdot x \cdot z_2 = y$ \\
                If $x = y$, then $z_1 = z_2 = \epsilon \iff x$ is a substring of $y$ and $y$ is a substring of $x$
                $z_1$ and $z_2$ can be $\epsilon$
                \item Lexigraphical Ordering over String: \\
                Requires order on $\Sigma$ can be used to order strings
            \end{itemize}
            \item Countable and Uncountable Sets
            \begin{itemize}
                \item $A: \{1, 2, 3\}$, $B: {x, y, z}$
                \item Function $f$ that maps $A$ to $B$ (one-to-one)
                \item $f(1) = x$, $f(2) = y$, $f(3) = z$
                \item $|A| = |B| = 3$ then $\exists f$ that is one-to-one correspondence
                \item Works for finite sets and countably infinite sets
                \item $\mathbb{N}$, $2\mathbb{N}$, $f: \mathbb{N} \rightarrow 2\mathbb{N}$
                \item If infinite set can be mapped to $\mathbb{N}$, then it is \underline{countably} infinite
                \item If infinite set cannot be mapped to $\mathbb{N}$, then it is \underline{uncountably} infinite
            \end{itemize}
        \end{itemize}
    \end{enumerate}
\end{document}