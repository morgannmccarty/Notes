\documentclass[12pt]{article}

\usepackage{hyperref}
\usepackage[margin=0.5in]{geometry}
\usepackage[fleqn]{amsmath}
\usepackage{amsfonts}
\usepackage{amsthm}
% \usepackage{tikz}

% chktex-file 8
% chktex-file 44

\title{
    Probability and Statistics:\ Lesson 4
    \\Continuous Random Variables}
\author{Morgan McCarty}
\date{11 July 2023}

\begin{document}
    \maketitle

    \section{Lesson 3 Addendum:\ Discrete Random Variables}
        \subsection{Definitions}
            \begin{itemize}
                \item \underline{Binomial Distribution}: $X \sim Binomial(n, p) = X_1 + X_2 + \cdots + X_n$ where $X_i \sim Bernoulli(p)$.
            \end{itemize}
        \subsection{Examples}
            \begin{enumerate}
                \item Flip a coin $n$ times
                \begin{itemize}
                    \item $X$ is the number of heads
                    \item $Ran(X) = \{0, 1, 2, \ldots, n\}$
                    \item pdf:
                    \begin{align*}
                        p_X(k) &= P(X = k) \\
                        p_X(0) &= (1-p^n) \\
                        p_X(1) &= \binom{n}{1} p^1 {(1-p)}^{n-1} \\
                        p_X(2) &= \binom{n}{2} p^2 {(1-p)}^{n-2} \\
                        p_X(k) &= \binom{n}{k} p^k {(1-p)}^{n-k}
                    \end{align*}
                    \item cdf:
                    \begin{align*}
                        F_X(k) &= P(X \leq k) \\
                        F_X(0) &= P(X = 0) = {(1-p)}^n \\
                        F_X(1) &= P(X \leq 1) = {(1-p)}^n + \binom{n}{1} p^1 {(1-p)}^{n-1} \\
                        F_X(2) &= P(X \leq 2) = {(1-p)}^n + \binom{n}{1} p^1 {(1-p)}^{n-1} + \binom{n}{2} p^2 {(1-p)}^{n-2} \\
                        F_X(k) &= \sum_{i=0}^{k} \binom{n}{i} p^i {(1-p)}^{n-i}
                    \end{align*}
                    \item Expected Value:
                    \begin{align*}
                        E[X] &= \mu = \sum_{x} x \cdot p_X(x) \\
                    \end{align*} 
                    \item Variance:
                    \begin{align*}
                        Var(X) &= \Sigma_{x} {(x - \mu)}^2 \cdot p_X(x) \\
                        &= E(x^2) - \mu^2 \\
                    \end{align*}
                \end{itemize}
            \end{enumerate}
    \section{Continuous Random Variables}
        \subsection{Definitions}
            \begin{itemize}
                \item \underline{Continuous Random Variable}: a random variable that can take on any value in an interval of the real numbers.
                \item \underline{Range of a Continuous Random Variable}: the set of all possible values of $X$. $Ran(X) = \{x \in \mathbb{R} | p_X(x) > 0\}$. A union of continuous intervals.
                \item \underline{Probability Density Function}: $f_X(x)$ is a function such that $f_X(x) \geq 0$ and $\int_{-\infty}^{\infty} f_X(x) dx = 1$.
                \item $\int_a^b f_X(x) dx = P(a \leq X \leq b)$: area under the curve between $a$ and $b$.
            \end{itemize}
        \subsection{Uniform Distribution}
            \begin{itemize}
                \item \underline{Uniform Distribution}: $X \sim Uniform(a, b)$ where $a < b$.
                \item $Ran(X) = (a, b)$
                \item pdf:
                \begin{align*}
                    f_X(x) &= c \\
                    \int_a^b f_X(x) dx &= 1 \\
                    \int_a^b c dx &= 1 \\
                    c(b-a) &= 1 \\
                    c &= \frac{1}{b-a} \\
                    f_X(x) &= \frac{1}{b-a}
                \end{align*}
                \item cdf:
                \begin{align*}
                    F_X(x) &= P(X \leq x) \\
                    &= \int_{-\infty}^{x} f_X(t) dt \\
                    &= \int_{a}^{x} \frac{1}{b-a} dt \\
                    &= \frac{t}{b-a} \Big|_{a}^{x} \\
                    &= \frac{x-a}{b-a}
                \end{align*}
                \item Expected Value:
                \begin{align*}
                    E[X] &= \int_{-\infty}^{\infty} x \cdot f_X(x) dx \\
                    &= \int_{a}^{b} x \cdot \frac{1}{b-a} dx \\
                    &= \frac{x^2}{2(b-a)} \Big|_{a}^{b} \\
                    &= \frac{b^2 - a^2}{2(b-a)} \\
                    &= \frac{b+a}{2}
                \end{align*}
                \item Variance:
                \begin{align*}
                    Var(X) &= E(X^2) - \mu^2 \\
                    &= \int_{-\infty}^{\infty} x^2 \cdot f_X(x) dx - \mu^2 \\
                    &= \int_{a}^{b} x^2 \cdot \frac{1}{b-a} dx - \mu^2 \\
                    &= \frac{x^3}{3(b-a)} \Big|_{a}^{b} - \mu^2 \\
                    &= \frac{b^3 - a^3}{3(b-a)} - \mu^2 \\
                    &= \frac{b^2 + ab + a^2}{3} - \mu^2 \\
                    &= \frac{b^2 + 2ab + a^2 - 3\mu^2}{3} \\
                    &= \frac{b^2 + 2ab + a^2 - 3\frac{{(b+a)}^2}{4}}{3} \\
                    &= \frac{4b^2 + 8ab + 4a^2 - 3b^2 - 6ab - 3a^2}{12} \\
                    &= \frac{b^2 - 2ab + a^2}{12} \\
                    &= \frac{{(b-a)}^2}{12} 
                \end{align*}
                \subsubsection{Note}
                    \begin{align*}
                        \frac{b^n-a^n}{b-a} &= \sum_{i=0}^{n-1} a^i b^{n-1-i}
                    \end{align*}
            \end{itemize}
        \subsection{Exponential Distributon}
            \begin{itemize}
                \item \underline{Exponential Distribution}: $X \sim Exponential(\lambda)$ where $\lambda > 0$.
                \item $Ran(X) = (0, \infty)$
                \item pdf:
                \begin{align*}
                    f_X(x) &= \lambda e^{-\lambda x} \\
                \end{align*}
                \item cdf:
                \begin{align*}
                    F_X(x) &= P(X \leq x) \\
                    &= \int_{-\infty}^{x} f_X(t) dt \\
                    &= \int_{0}^{x} \lambda e^{-\lambda t} dt \\
                    &= -e^{-\lambda t} \Big|_{0}^{x} \\
                    &= -e^{-\lambda x} + 1 \\
                    &= 1 - e^{-\lambda x}
                \end{align*}
                \item Expected Value:
                \begin{align*}
                    E(X) &= \int_{-\infty}^{\infty} x \cdot f_X(x) dx \\
                    &= \frac{1}{\lambda}
                \end{align*}
            \end{itemize}
\end{document}