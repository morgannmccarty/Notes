\documentclass[12pt]{article}

\usepackage{hyperref}
\usepackage[margin=0.5in]{geometry}
\usepackage{amsfonts}

\title{Probability and Statistics: Lesson 1}
\author{Morgan McCarty}
\date{03 July 2023}

\begin{document}
    \maketitle

    \section{General Overview}
        \subsection{Definitions}
            \underline{Experiment}: procedure with undetermined \underline{outcomes}
            \\\underline{Sample Space}: (S.S. or S) \underline{set} of all possible outcomes
            \\\underline{Set}: a collection of things
            \\\underline{Countable}: can be put in one-to-one correspondence with the natural numbers (integers are countable)
            \\\underline{Discrete}: finite or countable
            \\\underline{Continuous}: uncountable (in opposition to discrete)
            \\\underline{Universal Set}: set of all possible outcomes equivalent to the sample space in a Probability experiment

        \subsection{Symbols}
            \begin{itemize}
                \item $\in$: $x \in S$: $x$ is an element of $S$
                \item $\notin$: $x \notin S$ $x$ is not an element of $S$
            \end{itemize}
        
        \subsection{Examples}
            \begin{itemize}
                \item 
                    \begin{itemize}
                        \item Experiment: flip a coin
                        \item Sample Space: $\{H, T\}$
                        \\Sample Space is \textbf{finite}
                    \end{itemize}
                \item 
                    \begin{itemize}
                        \item Experiment: flip a coin until we get a tails
                        \item Sample Space: $\{T, HT, HHT, HHHT, \cdots\}$
                        \\Sample Space is \textbf{infinite}, but \textbf{countable}
                    \end{itemize}
                \item 
                    \begin{itemize}
                        \item Experiment: pick a number in the interval $[0, 1]$
                        \item Sample Space: $[0, 1]$ or $\{ x \in \mathbb{R} \mid 0 \leq x \leq 1 \}$
                        \\Sample Space is \textbf{infinite}, and \textbf{not countable}
                    \end{itemize}
            \end{itemize}
        
    \section{Events}
        \subsection{Definitions}
            \underline{Subset}: a set whose elements are all contained in another (super)set, additionally every set is a subset of itself and the empty set is a subset of every set
            \\\underline{Event}: a subset of the sample space
        \subsection{Symbols}
            \begin{itemize}
                \item $\subseteq$: $A \subseteq B$: $A$ is a subset of $B$
                \item $\subset$: $A \subset B$: $A$ is a proper subset of $B$ (at least one element of $B$ is not in $A$)
                \item $\emptyset$: the empty set
            \end{itemize}
        \subsection{Examples}
            \begin{itemize}
                \item 
                    \begin{itemize}
                        \item Roll a six-sided die
                        \item $S = \{1, 2, 3, 4, 5, 6\}$
                        \\Sample Space is \textbf{finite}
                        \item Events:
                            \begin{itemize}
                                \item Event of rolling even numbers: $A = \{2, 4, 6\}$
                                \item Event of rollng a ``$6$'': $B = \{6\}$
                                \item Event of rollling a prime number: $C = \{2, 3, 5\}$
                                \item Event of rolling a number $7$ or greater: $D = \emptyset$
                            \end{itemize}
                    \end{itemize}
            \end{itemize}
\end{document}