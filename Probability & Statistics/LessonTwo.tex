\documentclass[12pt]{article}

\usepackage{hyperref}
\usepackage[margin=0.5in]{geometry}
\usepackage[fleqn]{amsmath}
\usepackage{amsfonts}
\usepackage{amsthm}
% \usepackage{tikz}

\title{
    Probability and Statistics: Lesson 2
    \\Conditional Probability and Bayes Theorem}
\author{Morgan McCarty}
\date{05 July 2023}

\begin{document}
    \maketitle

    \section{Lesson 1 Review}
        \subsection{Review Definitions}
            \begin{enumerate}
                \item Sample Space (S): the set of all possible outcomes
                \item Events: a subset of the Sample Space - e.g. $A \subseteq S$
            \end{enumerate}
        \subsection{Review Set Operation Notation}
            For sets $A$ and $B \subseteq S$:
            \begin{enumerate}
                \item Complement: $A^c = \{x \in S | x \notin A\}$
                \item Intersection: $A \cap B = \{x \in S | x \in A \text{ and } x \in B\}$
                \item Union: $A \cup B = \{x \in S | x \in A \text{ or } x \in B\}$
            \end{enumerate}
        \subsection{Probability Function}
            \subsubsection{Informally}
                $P(A)$ is the probability that event $A$ occurs with output in the range $[0, 1]$
            \subsubsection{Formally}
                For a function $f$ the domain is the set of all possible inputs ($A$) and the codomain is the set of all possible outputs ($B$)
                \\i.e. $f: A \rightarrow B$
                \\For a probability function $P$ the domain is the power set (the set of all subsets of a set) of the Sample Space ($\mathcal{P}(S)$) and the codomain is the set of all real numbers in the range $[0, 1]$
                \\i.e. $P: \mathcal{P}(S) \rightarrow [0, 1]$
        \subsection{Axioms and Properties of a Probability Function}
            Given some probabilities, it is possible to calculate other probabilities
            \subsubsection{Basic Principle of Probability}
                Assuming equally likely outcomes, the probability of an event $A$ is:
                \begin{equation}
                    P(A) = \frac{\text{number of outcomes favorable to } A}{\text{total number of possible outcomes}}
                \end{equation}
                or
                \begin{equation}
                    P(A) = \frac{|A|}{|S|}
                \end{equation}
        \subsection{Counting}
            There are two main rules of counting, the addition and multiplication rules which are mostly self-explanatory
    \section{Conditional Probability (of event $A$ given event $B$ [has occurred])}
        [$S$ is our Sample Space; $A$ and $B$ are events in $S$]
        \subsection{Basic Notation}
            \begin{equation}
                P(A|B) = \frac{P(A \cap B)}{P(B)}
            \end{equation}
            This is the probability of event $A$ given that event $B$ has occurred
        \subsection{Definitions}
            \begin{enumerate}
                \item Contracted Sample Space: $B$ is the new Sample Space as if $B$ has occurred only events in $B$ are possible
                \item Product Law: $P(A \cap B) = P(A|B)P(B)$
            \end{enumerate}
        \subsection{$P(A|B)$ in Prior Terms}
            \begin{equation}
                P(A|B) = \frac{P(A \cap B)}{P(B)} = \frac{\frac{|A \cap B|}{|S|}}{\frac{|B|}{|S|}} = \frac{|A \cap B|}{|B|}
            \end{equation}
        \subsection{Examples}
            \begin{enumerate}
                \item In a town of people, among all voters during a recent election, how many are first-time voters who voted for ``Bob''?
                \begin{enumerate}
                    \item $S$ is the set of all voters
                    \item $A$ is the set of all first-time voters
                    \item $B$ is the set of all voters who voted for ``Bob''
                    \item $A \cap B^c = 260$
                    \item $A \cap B = 465$ 
                    \item $A^c \cap B = 1123$
                    \item $A^c \cap B^c = 1507$
                    \item $P(B) = \frac{|B|}{|S|} = \frac{465 + 1123}{260 + 465 + 1123 + 1507} = \frac{1588}{3355}$
                    \item $P(B|A) = \frac{|A \cap B|}{|A|} = \frac{465}{260 + 465} = \frac{465}{725}$
                \end{enumerate}
                \item In a bag of $9$ chips: $3$ are \textbf{red} and $6$ are \textbf{white}.
                \item Chips are drawn at random:
                \begin{enumerate}
                    \item If a \textbf{red} chip is drawn, it is placed back in the bag.
                    \item If a \textbf{white} chip is drawn, it is placed back in the bag along with a new \textbf{red} chip.
                \end{enumerate}
                \item What is the probability that, if two chips are drawn, both are \textbf{red}?
                \begin{enumerate}
                    \item $A$ is the event that the first chip is \textbf{red}
                    \item $B$ is the event that the second chip is \textbf{red}
                    \item $P(A) = \frac{3}{9} = \frac{1}{3}$
                    \item $P(A^c) = \frac{6}{9} = \frac{2}{3}$
                    \item $P(B|A) = \frac{4}{10}$
                    \item $P(B^c|A) = \frac{6}{10}$
                    \item $P(B|A^c) = \frac{3}{9}$
                    \item $P(B^c|A^c) = \frac{6}{9}$
                    \item Therefore $P(A \cap B)$ (the probability that both chips are \textbf{red}) $= P(A)P(B|A) = \frac{3}{9}\times\frac{4}{10} = \frac{2}{15}$
                \end{enumerate}
                \item What is the probability the second chip is \textbf{red}? (i.e. $P(B)$)
                \begin{align*}
                    B &= (A \cap B) \sqcup (A^c \cap B)\\
                    P(B) &= P(A \cap B) + P(A^c \cap B)\\
                    &= P(A)P(B|A) + P(A^c)P(B|A^c)\\
                    &= \frac{3}{9}\times\frac{4}{10} + \frac{6}{9}\times\frac{3}{9}\\
                    &= \frac{2}{15} + \frac{2}{9}\\
                    &\approx 0.3556
                \end{align*}
            \end{enumerate}
        \subsection{Law of Total Probability}
            This generalizes to:
            \begin{align*}
                P(B) &= P(A_1 /cap B) + P(A_2 \cap B) + \cdots + P(A_n \cap B)\\
                &= P(A_1)P(B|A_1) + P(A_2)P(B|A_2) + \cdots + P(A_n)P(B|A_n)
            \end{align*}
            or
            \begin{equation}
                P(B) = \sum_{i=1}^n P(A_i)P(B|A_i)
            \end{equation}
            This is a weighted average of the conditional probabilities of $B$ given $A_i$ for all $i$.
        \subsection{Examples}
            \begin{enumerate}
                \item There are $3$ suppliers of computer chips.
                \\Supplier $A$ supplies $50\%$ of the chips, supplier $B$ supplies $40\%$, and supplier $C$ supplies $10\%$.
                \\Chips from supplier $A$ fail $1\%$ of the time, chips from supplier $B$ fail $2\%$ of the time, and chips from supplier $C$ fail $5\%$ of the time.
                \\If a random chip is chosen, what is the probability that it will fail? (i.e. $P(F)$)
                \begin{enumerate}
                    \item $A$ is the event that the chip is from supplier $A$
                    \item $B$ is the event that the chip is from supplier $B$
                    \item $C$ is the event that the chip is from supplier $C$
                    \item $F$ is the event that the chip fails
                    \item $P(A) = 0.5$ and $P(B) = 0.4$ and $P(C) = 0.1$
                    \item $P(F|A) = 0.01$ and $P(F^c|A) = 0.99$
                    \item $P(F|B) = 0.02$ and $P(F^c|B) = 0.98$
                    \item $P(F|C) = 0.05$ and $P(F^c|C) = 0.95$
                    \item Therefore $P(F) = P(A)P(F|A) + P(B)P(F|B) + P(C)P(F|C) = 0.5\times0.01 + 0.4\times0.02 + 0.1\times0.05 = 0.018$
                \end{enumerate}
                \item In a standard deck of cards where two cards are drawn sequentially without being revealed:
                \begin{enumerate}
                    \item What is the probability that the first card is a \textbf{heart}?
                    \\$P(A) = \frac{13}{52} = \frac{1}{4}$
                    \item What is the probability that the second card is a \textbf{heart} given that the first card is a \textbf{heart}?
                    \\$P(B|A) = \frac{12}{51}$
                    \item What is the probability that the second card is a \textbf{heart}?
                    \\$P(B) = P(A)P(B|A) + P(A^c)P(B|A^c) = \frac{1}{4}\times\frac{12}{51} + \frac{3}{4}\times\frac{13}{51} = \frac{1}{4}$
                    \\``What we don't know doesn't matter'', i.e. if we do not know whether the first card is a \textbf{heart} or not, the probability that the second card is a \textbf{heart} is still $\frac{1}{4}$
                \end{enumerate}
            \end{enumerate}
    \section{Bayes' Theorem}
        \subsection{What is the Inverted Conditional Probability?}
            i.e. given $P(B|A)$, what is $P(A|B)$?
            \begin{align*}
                P(A|B) &= \frac{P(A \cap B)}{P(B)}\\
                &= \frac{P(B|A)P(A)}{P(B)}\\
                &= \frac{P(B|A)P(A)}{P(B|A)P(A) + P(B|A^c)P(A^c)}
            \end{align*}
        \subsection{Definition}
            \begin{equation}
                P(A|B) = \frac{P(B|A)P(A)}{P(B|A)P(A) + P(B|A^c)P(A^c)}
            \end{equation}
            or
            \begin{equation}
                P(A|B) = \frac{P(B|A)P(A)}{P(B)}
            \end{equation}
        \subsection{Examples}
            \begin{enumerate}
                \item Continuing the chip supplier example: if a chip fails, what is the probability that it came from supplier $A$?
                \begin{enumerate}
                    \item We need to find $P(A|F)$
                    \begin{align*}
                        P(A|F) &= \frac{P(F|A)P(A)}{P(F)}\\
                        &= \frac{0.01\times0.5}{.018}\\
                        &= \frac{0.005}{0.018}\\
                        &= \frac{5}{18}
                    \end{align*}
                \end{enumerate}
                \item If someone is asked whether they smoke and they say they do not, what is the probability they actually do smoke?
                \\$S$ is the event that the person smokes
                \\$N$ is the event that the person says they do not smoke
                \\$P(S) = 0.25$, $P(N|S) = 0.3$, and $P(N|S^c) = 1$
                \begin{enumerate}
                    \item We need to find $P(S|N)$
                    \begin{align*}
                        P(S|N) &= \frac{P(S \cap N)}{P(N)}\\
                        &= \frac{P(N|S)P(S)}{P(N)}\\
                        &= \frac{0.25\times0.3}{0.25\times0.3+0.75\times1}\\
                        &= \frac{0.075}{0.075+0.75}\\
                        &= \frac{0.075}{0.825}\\
                        &= \frac{1}{11}
                    \end{align*}
                    \item We can also find $P(S^c|N)$
                    \begin{align*}
                        P(S^c|N) &= \frac{P(S^c \cap N)}{P(N)}\\
                        &= \frac{P(N|S^c)P(S^c)}{P(N)}\\
                        &= \frac{0.75\times1}{0.25\times0.3+0.75\times1}\\
                        &= \frac{0.75}{0.075+0.75}\\
                        &= \frac{0.75}{0.825}\\
                        &= \frac{10}{11}
                    \end{align*}
                \end{enumerate} 
            \end{enumerate}
            
            
\end{document}