\documentclass[12pt]{article}

\usepackage{hyperref}
\usepackage[margin=0.5in]{geometry}
\usepackage[fleqn]{amsmath}
\usepackage{amsfonts}
\usepackage{amsthm}
% \usepackage{tikz}

% chktex-file 8
% chktex-file 44

\title{
    Probability and Statistics: Lesson 3
    \\Random Variables: Discrete}
\author{Morgan McCarty}
\date{10 July 2023}

\begin{document}
    \maketitle

    \section{Lesson 1 Addendum}
        \subsection{Counting}
            \subsubsection{Defiinitions}
                \begin{itemize}
                    \item \underline{Combination}: $nCk$ or $\binom{n}{k}$ is the number of ways to choose $k$ objects from a set of $n$ objects. \\
                    \begin{equation}
                        \binom{n}{k} = \frac{n!}{k!(n-k)!} = \frac{nPk}{k!}
                    \end{equation}
                    \item \underline{Permutation}: $nPk$ is the number of ways to arrange $k$ objects from a set of $n$ objects.
                    \begin{equation}
                        nPk = \frac{n!}{(n-k)!} = k!\binom{n}{k}
                    \end{equation}
                \end{itemize}
            \subsubsection{Examples}
                \begin{itemize}
                    \item Given the set $S = \{A, B, C\}$ how many ways can we choose $3$ letters from $S$? How many ways can we arrange those letters?
                    \begin{itemize}
                        \item $3C3 = 1$
                        \item $3P3 = 6$
                    \end{itemize}
                    \item There is only one way to choose $3$ letters from $S$ because there are only $3$ letters in $S$. However, there are $6$ ways to arrange those letters.
                \end{itemize}
    \section{Random Variable: Discrete}
        \subsection{Definition}
            \begin{itemize}
                \item \underline{Random Variable}: a function that maps the sample space to the real numbers. $X: S \rightarrow \mathbb{R}$
                \item \underline{Probability Density Functon} or \underline{Probability Distributon}: $p_X(x) = P(X = x)$ or $p_X(k) = P(X = k)$. Often written as a table.
                \item \underline{Cumulative Distribution Function}: $F_X(x) = P(X \leq x)$ or $F_X(k) = P(X \leq k)$. Often written as a table. \\
                \begin{equation}
                    F_X(x) = \sum_{k \leq x} p_X(k)
                \end{equation}
                \item \underline{Expected Value}: $E[X] = \mu_X = \sum_{k} k \cdot p_X(k)$
                \item \underline{Range}: the set of all possible values of $X$. $Ran(X) = \{x \in \mathbb{R} | p_X(x) > 0\}$.
                \item \underline{Variance}: $Var(X) = \sigma_X^2 = E[{(X - \mu_X)}^2] = \sum_{k} {(k - \mu_X)}^2 \cdot p_X(k)$ \\
                The useful equation $Var(X) = E(X^2) - \mu^2$ is derived in the proof below. \\
                Proof of $Var(X) = E(X^2) - \mu^2$:
                \begin{align*}
                    Var(X) = E({(X-\mu)}^2)
                    &= E(X^2 - 2X\mu + \mu^2) \\
                    &= E(X^2) - 2\mu{}E(X) + E(\mu^2) \\
                    &= E(X^2) - 2\mu^2 + \mu^2 \\
                    &= E(X^2) - \mu^2 \\
                \end{align*}
                The Variance is the expected value of the squared difference between $X$ and the expected value of $X$. \\
                In layman's terms, the variance is the average squared distance from the mean.
                \item \underline{Standard Deviation}: $\sigma_X = \sqrt{Var(X)}$ \\
                The standard deviation is the square root of the variance or the average distance from the mean.
            \end{itemize}
        \subsection{Examples}
            \begin{enumerate}
                \item Example: $2$ Spinning Wheels
                \begin{itemize}
                    \item Given $2$ $3$-way spinning wheels, find the probability distribution of the sum of the two wheels: $X$.
                    \begin{align*}
                        S = \begin{Bmatrix}
                            (1, 1), & (1, 2), & (1, 3), \\
                            (2, 1), & (2, 2), & (2, 3), \\
                            (3, 1), & (3, 2), & (3, 3)\  \\
                        \end{Bmatrix}
                    \end{align*}
                    \item The probability distribution of $X$ is:
                    \begin{align*}
                        \begin{array}{c|c|c|c|c|c|c|c|c|c}
                            k & 2 & 3 & 4 & 5 & 6 \\
                            \hline
                            p_X(k) & \frac{1}{9} & \frac{2}{9} & \frac{3}{9} & \frac{2}{9} & \frac{1}{9} \\
                            \hline
                        \end{array}
                    \end{align*}
                    \item The cumulative distribution of $X$ is:
                    \begin{align*}
                        \begin{array}{c|c|c|c|c|c|c|c|c|c}
                            k & 2 & 3 & 4 & 5 & 6 \\
                            \hline
                            F_X(k) & \frac{1}{9} & \frac{3}{9} & \frac{6}{9} & \frac{8}{9} & \frac{9}{9} \\
                            \hline
                        \end{array}
                    \end{align*}
                    \item The expected value of $X$ is:
                    \begin{align*}
                        E(X) &= \mu_X \\
                        &= \sum_{k} k \cdot p_X(k) \\ 
                        &= (2 \cdot \frac{1}{9}) + (3 \cdot \frac{2}{9}) + (4 \cdot \frac{3}{9}) + (5 \cdot \frac{2}{9}) + (6 \cdot \frac{1}{9}) \\
                        &= \frac{(2 + 6 + 12 + 10 + 6)}{9} \\
                        &= \frac{36}{9} \\
                        &= 4 \\
                    \end{align*}
                    \item The variance of $X$ is:
                    \begin{align*}
                        Var(X) &= \sigma_X^2 \\
                        &= E(X^2) - \mu^2 \\
                        &= 2^2 \cdot \frac{1}{9} + 3^2 \cdot \frac{2}{9} + 4^2 \cdot \frac{3}{9} + 5^2 \cdot \frac{2}{9} + 6^2 \cdot \frac{1}{9} - 4^2 \\
                        &= \frac{(4 + 18 + 48 + 50 + 36)}{9} - 16 \\
                        &= \frac{156}{9} - 16 \\
                        &= \frac{12}{9} \\
                        &= \frac{4}{3} \\
                    \end{align*}
                    \item The standard deviation of $X$ is:
                    \begin{align*}
                        \sigma_X &= \sqrt{Var(X)} \\
                        &= \sqrt{\frac{4}{3}} \\
                        &= \frac{2}{\sqrt{3}} \\
                    \end{align*}
                \end{itemize}
                \item Example: $3$ Fair Dice
                \begin{itemize}
                    \item Given $3$ fair dice, $X$ is the largest number rolled.
                    \item While usually, the probability density function is easier to find than the cumulative distribution function, in this case, the opposite is true.
                    \item Additional both can be written as equations rather than tables.
                    \item The cumulative distribution function of $X$ is:
                    \begin{align*}
                        F_X(k) = \frac{k^3}{6^3}
                    \end{align*}
                    \item The probability density function of $X$ is:
                    \begin{align*}
                        p_X(k) = F_X(k) - F_X(k-1)
                    \end{align*}
                \end{itemize}
            \end{enumerate}
        \subsection{Bernouli Variable (Distribution)}
            \begin{itemize}
                \item \underline{Bernouli Variable}: a random variable that can only take on two values, $0$ or $1$.
                \item \underline{Bernouli Distribution}: the probability distribution of a Bernouli Variable.
                \begin{align*}
                    X \sim Bernouli(p)
                \end{align*}
                \begin{align*}
                    p_X(k) = \begin{cases}
                        p & k = 1 \\
                        1 - p & k = 0
                    \end{cases}
                \end{align*}
                \begin{align*}
                    F_X(k) = \begin{cases}
                        0 & k < 0 \\
                        1 - p & 0 \leq k < 1 \\
                        1 & k \geq 1
                    \end{cases}
                \end{align*}
                \begin{align*}
                    E(X) &= 0 \cdot (1 - p) + 1 \cdot p \\
                    &= p
                \end{align*}
                \begin{align*}
                    Var(X) &= E(X^2) - \mu^2 \\
                    &= 0^2 \cdot (1 - p) + 1^2 \cdot p - p^2 \\
                    &= p - p^2 \\
                    &= p(1 - p)
                \end{align*}
            \end{itemize}
\end{document}